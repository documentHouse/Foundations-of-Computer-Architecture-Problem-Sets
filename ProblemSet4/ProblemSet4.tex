%\documentclass[11pt,reqno]{amsart}
\documentclass[11pt,reqno]{article}
\usepackage[margin=.8in, paperwidth=8.5in, paperheight=11in]{geometry}
%\usepackage{geometry}                % See geometry.pdf to learn the layout options. There are lots.
%\geometry{letterpaper}                   % ... or a4paper or a5paper or ... 
%\geometry{landscape}                % Activate for for rotated page geometry
%\usepackage[parfill]{parskip}    % Activate to begin paragraphs with an empty line rather than an indent7
\usepackage{array}
\usepackage{graphicx}
\usepackage{pstricks}
\usepackage{amssymb}
\usepackage{epstopdf}
\usepackage{amsmath}
\usepackage{subfigure}
\usepackage{caption}
\pagestyle{plain}
%\renewcommand{\topfraction}{0.3}
%\renewcommand{\bottomfraction}{0.8}
%\renewcommand{\textfraction}{0.07}
\DeclareGraphicsRule{.tif}{png}{.png}{`convert #1 `dirname #1`/`basename #1 .tif`.png}

\title{Foundations of Computer Architecture: \\ Problem Set 4 }
\author{Andrew Rickert}
\date{Professor: Dr. Horace Malcolm \\ \hspace{-19pt} Due Date: February 28,  2012}                                           % Activate to display a given date or no date

\begin{document}
\maketitle


% Page 1
\begin{flushleft} 
Problem 1 \\
\rule{500pt}{1pt}\\
\end{flushleft} 

The table below describes the truth values for $\overline{X \oplus Y}$ that we use to derive the logical sum of logical products.\\

\setlength{\extrarowheight}{2.0pt}
\begin{tabular}{| c | c | c | c |}
\hline
X & Y & X $\oplus$ Y & $\overline{X \oplus Y}$ \\ \hline
0 & 0 & 0 & 1 \\ \hline
0 & 1 & 1 & 0 \\ \hline
1 & 0 & 1 & 0 \\ \hline
1 & 1 & 0 & 1 \\ \hline
\end{tabular}\\ \\

\noindent The 1 terms of $\overline{X \oplus Y}$ are when $X = Y = 0$ and $X = Y = 1$. This leads to the form
\[ \overline{X \oplus Y} = \overline{X} \cdot \overline{Y} + X \cdot Y \]
\newpage

\begin{flushleft} 
Problem 2 \\
\rule{500pt}{1pt}\\
\end{flushleft} 

\noindent\framebox{Part a}\\ 

The table below gives the full adder sum $S$ which is composed of operands $X,Y$ and the carry-in value $CI$.\\

\setlength{\extrarowheight}{0pt}
\begin{tabular}{| c | c | c | c |}
\hline
X & Y & CI & S \\ \hline
0 & 0 & 0 & 0 \\ \hline
1 & 0 & 0 & 1 \\ \hline
0 & 1 & 0 & 1 \\ \hline
1 & 1 & 0 & 0 \\ \hline
0 & 0 & 1 & 1 \\ \hline
1 & 0 & 1 & 0 \\ \hline
0 & 1 & 1 & 0 \\ \hline
1 & 1 & 1 & 1 \\ \hline
\end{tabular}\\ \\

\noindent\framebox{Part b}\\ 

\noindent If we let $\land$ denote the xor function then we have the following is the truth table for $X \land Y \land CI$.\\

\begin{tabular}{| c | c | c | c |}
\hline
X & Y & CI & $X \land Y \land CI$ \\ \hline
0 & 0 & 0 & 0 \\ \hline
1 & 0 & 0 & 1 \\ \hline
0 & 1 & 0 & 1 \\ \hline
1 & 1 & 0 & 0 \\ \hline
0 & 0 & 1 & 1 \\ \hline
1 & 0 & 1 & 0 \\ \hline
0 & 1 & 1 & 0 \\ \hline
1 & 1 & 1 & 1 \\ \hline
\end{tabular}\\ \\

\noindent The $\land$ table results in the same values as the sum $S$ in the previous part.
\newpage

\begin{flushleft} 
Problem 3 \\
\rule{500pt}{1pt}\\
\end{flushleft} 

To answer the question first an expression for the output of the circuit is derived then the expression is simplified.\\
There are three $AND$ gates lined up vertically in the center of the drawing. The first of these $AND$ gates (the top most) will produce the term $\overline{A \cdot B} \cdot E$. The second $AND$ gate, or middle gate will produce the term $E \cdot F$. The final $AND$ gate, the one at the bottom will produce the term $\overline{C + D} \cdot F$. The final expression is a $NOR$ of these terms which gives the final unsimplified value $V$ as:
\[ V = \overline{ \overline{A \cdot B} \cdot E + E \cdot F + \overline{C + D} \cdot F } \]

\noindent We know show the simplification derivation:
\begin{eqnarray*}
 V &=& \overline{ \overline{A \cdot B} \cdot E + E \cdot F + \overline{C + D} \cdot F } \\
     &=& \overline{ (\overline{A \cdot B} + F ) \cdot E + \overline{C + D} \cdot F } \qquad \text{By the distributive law}\\
     &=& \overline{(\overline{A \cdot B} + F ) \cdot E} \cdot \overline {\overline{C + D} \cdot F} \qquad \text{By de morgans law}\\
     &=& (\overline{(\overline{A \cdot B} + F )} + \overline{E}) \cdot (\overline {\overline{C + D}} + \overline{F}) \qquad \text{By de morgans law}\\
     &=& (\overline{\overline{A \cdot B}} \cdot \overline{F} + \overline{E}) \cdot (\overline {\overline{C + D}} + \overline{F}) \qquad \text{By de morgans law}\\
     &=& (A \cdot B \cdot \overline{F} + \overline{E}) \cdot (C + D + \overline{F}) \qquad \text{By double negation}\\
\end{eqnarray*}

\noindent This is expression $b$. We may use de morgans law to derive another equivalent expression
\begin{eqnarray*}
 V &=& (A \cdot B \cdot \overline{F} + \overline{E}) \cdot (C + D + \overline{F}) \\
     &=& \overline{(A \cdot B \cdot \overline{F} + \overline{E})} + \overline{(C + D + \overline{F})} \ \qquad \text{By de morgans law} \\
     &=& \overline{(A \cdot B \cdot \overline{F} + \overline{E})} + \overline{(C + D + \overline{F})} \ \qquad \text{By de morgans law} \\
     &=&\overline{(A \cdot B \cdot \overline{F}} \cdot \overline{\overline{E}}) + \overline{C + D} \cdot \overline{\overline{F}} \ \qquad \text{By de morgans law} \\
     &=&\overline{(A \cdot B \cdot \overline{F}} \cdot E) + \overline{C} \cdot \overline{D} \cdot F \ \qquad \text{By de morgans law and double negation} \\
     &=&(\overline{A \cdot B} + \overline{\overline{F}}) \cdot E + \overline{C} \cdot \overline{D} \cdot F \ \qquad \text{By de morgans law} \\
     &=&(\overline{A} + \overline{B} + F) \cdot E + \overline{C} \cdot \overline{D} \cdot F \ \qquad \text{By de morgans law and double negation} \\
     &=&(\overline{A} + \overline{B}) \cdot E + F \cdot E + \overline{C} \cdot \overline{D} \cdot F \ \qquad \text{By de morgans law and double negation} \\
\end{eqnarray*}

\noindent This is expression $e$. So, the output generated by the circuit is described by both expressions $b$ and $e$.
\newpage

\begin{flushleft} 
Problem 4 \\
\rule{500pt}{1pt}\\
\end{flushleft} 

\noindent The table for the function $F(x,y,z)$ is copied below for reference:\\

\begin{tabular}{| l | c | c | c | c |}
\hline
Row & x & y & z & F \\ \hline
1 & 0 & 0 & 0 & 0 \\ \hline
2 & 1 & 0 & 1 & 0 \\ \hline
3 & 0 & 1 & 0 & 1 \\ \hline
4 & 1 & 1 & 1 & 0 \\ \hline
5 & 0 & 0 & 0 & 0 \\ \hline
6 & 1 & 0 & 1 & 1 \\ \hline
7 & 0 & 1 & 0 & 1 \\ \hline
8 & 1 & 1 & 1 & 1 \\ \hline
\end{tabular}\\ \\

\noindent\framebox{Part a}\\ 

\noindent The nonzero minterms occur in rows 3,6,7,8. The minterms for these rows are:\\

\begin{tabular}{| l | c |}
\hline
Row & minterm \\ \hline
3 & $\overline{x} \cdot y \cdot \overline{z}$ \\ \hline
6 & $x \cdot \overline y \cdot z$ \\ \hline
7 & $\overline{x} \cdot y \cdot \overline{z}$ \\ \hline
8 & $x \cdot y \cdot z$ \\ \hline
\end{tabular}\\ \\

\noindent Note: row 3 produces the same minterm as row 6, so there are only 3 unique minterms.\\

\noindent\framebox{Part b}\\ 

\noindent The zero maxterms occur in rows 1,2,4,5. The maxterms for these rows are:\\

\begin{tabular}{| l | c |}
\hline
Row & maxterm \\ \hline
1 & $x + y + z$ \\ \hline
2 & $\overline{x} + y + \overline{z}$ \\ \hline
4 & $\overline{x} + \overline{y} + \overline{z}$ \\ \hline
5 & $x + y + z$ \\ \hline
\end{tabular}\\ \\

\noindent Note: row 1 produces the same maxterm as row 5, so there are only 3 unique maxterms.\\
\newpage

\begin{flushleft} 
Problem 5 \\
\rule{500pt}{1pt}\\
\end{flushleft} 

\noindent The first two parts apply to the first circuit. The second two parts apply to the second circuit.\\

\noindent\framebox{Part a}\\ 

The complete table for the circuit is below:\\

\begin{tabular}{| c | c | c |}
\hline
A & B & C \\ \hline
0 & 0 & 1 \\ \hline
0 & 1 & 0 \\ \hline
1 & 0 & 1 \\ \hline
1 & 1 & 1 \\ \hline
\end{tabular}\\ \\

\noindent\framebox{Part b}\\ 

Using the table in part $a$ we can use the zero maxterms to define the function
\[ C = f(A,B) = A + \overline{B}\]

\noindent\framebox{Part c}\\ 


The complete table for the circuit is below:\\

\begin{tabular}{| c | c | c |}
\hline
A & B & C \\ \hline
0 & 0 & 0 \\ \hline
0 & 1 & 1 \\ \hline
1 & 0 & 0 \\ \hline
1 & 1 & 0 \\ \hline
\end{tabular}\\ \\

\noindent\framebox{Part d}\\ 

Using the table in part $c$ we can use the zero maxterms to define the function
\[ C = f(A,B) = \overline{A} \cdot B\]

\end{document}  