%\documentclass[11pt,reqno]{amsart}
\documentclass[11pt,reqno]{article}
\usepackage[margin=.8in, paperwidth=8.5in, paperheight=11in]{geometry}
%\usepackage{geometry}                % See geometry.pdf to learn the layout options. There are lots.
%\geometry{letterpaper}                   % ... or a4paper or a5paper or ... 
%\geometry{landscape}                % Activate for for rotated page geometry
%\usepackage[parfill]{parskip}    % Activate to begin paragraphs with an empty line rather than an indent7
\usepackage{array}
\usepackage{graphicx}
\usepackage{pstricks}
\usepackage{amssymb}
\usepackage{epstopdf}
\usepackage{amsmath}
\usepackage{subfigure}
\usepackage{caption}
\pagestyle{plain}
%\renewcommand{\topfraction}{0.3}
%\renewcommand{\bottomfraction}{0.8}
%\renewcommand{\textfraction}{0.07}
\DeclareGraphicsRule{.tif}{png}{.png}{`convert #1 `dirname #1`/`basename #1 .tif`.png}

\title{Foundations of Computer Architecture: \\ Problem Set 8 }
\author{Andrew Rickert}
\date{Professor: Dr. Horace Malcolm \\ \hspace{-46pt} Due Date: April 3,  2012}                                           % Activate to display a given date or no date

\begin{document}
\maketitle


% Page 1
\begin{flushleft} 
Problem 1 \\
\rule{500pt}{1pt}\\
\end{flushleft} 

\noindent\framebox{Part a}\\ 

For the string "Spring 2014" the first character in the big endian format would be '4', then '1' and so on. The characters at addresses ending in 4,5,6 and B will be '4','1','0' and 'i' respectively. These values are converted to ASCII then entered into the table below in hexadecimal notation.\\

\begin{tabular}{| c | c |}
\hline
Address & Contents \\ \hline
0x100C004 & 0x34\\ \hline
0x100C005 & 0x31\\ \hline
0x100C006 & 0x30\\ \hline
0x100C00B & 0x69\\ \hline
\end{tabular}
\vspace{10pt}

\noindent\framebox{Part b}\\ 

For the string "Spring 2014" the first character in the little endian format would be 'S', then 'p' and so on. The characters at addresses ending in 4,5,6 and B will be 'S','p','r' and '2' respectively. These values are converted to ASCII then entered into the table below in hexadecimal notation.\\

\begin{tabular}{| c | c |}
\hline
Address & Contents \\ \hline
0x100C004 & 0x53\\ \hline
0x100C005 & 0x70\\ \hline
0x100C006 & 0x72\\ \hline
0x100C00B & 0x32\\ \hline
\end{tabular}
\newpage

\begin{flushleft} 
Problem 2 \\
\rule{500pt}{1pt}\\
\end{flushleft} 

The table below contains the contents of the register $\$t0$ for load instruction in the big and little endian formats:\\

\begin{tabular}{| c | c | c |}
\hline
& Big Endian & Little Endian \\ \hline
lw \$t0,0(\$t1) & 0xA987D543 & 0x43D587A9 \\ \hline
lhu \$t0,2(\$t1) & 0x0000D543 & 0x000043D5 \\ \hline
lb \$t0,3(\$t1) & 0x00000043 & 0x00000043 \\ \hline
lbu \$t0,0(\$t1) & 0x000000A9 & 0x000000A9 \\ \hline
lb \$t0,1(\$t1) & 0xFFFFFF87 & 0xFFFFFF87 \\ \hline
lh \$t0,2(\$t1) & 0xFFFFD543 & 0x000043D5\\ \hline
\end{tabular}
\newpage

\begin{flushleft} 
Problem 3 \\
\rule{500pt}{1pt}\\
\end{flushleft} 

\noindent\framebox{Part a}\\ 

The total number of bytes in the system is $128 \times 2^{20} = 2^{27}$ bytes. The number of bytes in a module is $134217728/8 = 16777216$. This gives the number of modules as $2^{27}/16777216 = 8$.\\

\noindent\framebox{Part b}\\ 

From the previous part of the problem we derived $D = 134217728/W = 134217728/8 = 16777216$.\\

\noindent\framebox{Part c}\\ 

The MAR's in this memory system are 32-bits wide. We have calculated there to be 8 separate modules to address. To locate one of the 8 modules we need 3 bits from an assumed 32 bit address bus. Because we are considering a high order interleaving we have the following address format:\\

\begin{tabular}{| c | c |}
\hline
XXX & XXXXX XXXXX XXXXX XXXXX XXXXX XXXX \\ \hline
module & offset \\ \hline
\end{tabular}\\

However since each module only has 16777216 bytes we only need 24 bits to represent the offset since $2^{24} = 16777216$. This would give the address format as:\\

\begin{tabular}{| c | c |}
\hline
XXX & XXXXX XXXXX XXXXX XXXXX XXXX \\ \hline
module & offset \\ \hline
\end{tabular}\\
\vspace{10pt}

\noindent\framebox{Part d}\\ 

\noindent For lower order interleaving in our 8 module system with a 32-bit bus we have the following address format:\\

\begin{tabular}{| c | c |}
\hline
XXXX XXXXX XXXXX XXXXX XXXXX XXXXX & XXX \\ \hline
module & offset \\ \hline
\end{tabular}
\vspace{10pt}\\
Since we only need 24 bits to represent the offset we could have the following address format:\\

\begin{tabular}{| c | c |}
\hline
XXXX XXXXX XXXXX XXXXX XXXXX & XXX\\ \hline
module & offset \\ \hline
\end{tabular}\\
\vspace{10pt}

\noindent\framebox{Part e}\\ 

As shown in the previous parts we need 3 lines to select the module. We would also need at minimum 24 lines to select the cell if we were not assuming word aligned 4 byte accesses. Since we are assuming the addressing is word aligned we can remove two of the lines and logical shift the value to get the offset. This gives a total of $3 + 24 - 2 = 25$ required lines.\\

\noindent\framebox{Part f}\\ 

If we could not be certain that the memory element we are looking for is on the word boundary then we would need to have the possibility of addressing non-4-byte word aligned addresses. This would mean that would could not use the optimization of removing two lines from the address bus which would \emph{increase} the size of the address bus.\\

\noindent\framebox{Part g}\\ 

Since we have 8 modules to query we use the first three bits of the address for low order interleaving. The lower three bits of $0x00040084$ are 100 which represents a 4. So this address will be in the 5th module (if we call the module 0 the first module).\\

\noindent\framebox{Part h}\\ 

Since we have 8 modules to query we use the last three bits of the address for low order interleaving. The lower three bits of $0x00040084$ are 000 which represents a 0. So this address will be in the 1st module (if we call the module 0 the first module).\\

\noindent\framebox{Part i}\\ 

The answer depends on whether sequential accesses to data are common or not. In the case that they are then a high order interleaved memory might not be a good solution. This is because many accesses would have to be made to the same module which might slow down processing.

\end{document}  